\begin{enumerate}[label=\thesection.\arabic*,ref=\thesection.\theenumi]
\item An LC circuit contains a $50 \mu H$ inductor and a $50 \mu F$ capacitor with an initial charge of $10 mC$. The resistance of the circuit is negligible. Let the instant the circuit is closed by $t = 0$.

\textbf{a)} What is the total energy stored initially? Is it conserved during LC oscillations?

\textbf{b)} What is the natural frequency of the circuit?

\textbf{c)} At what time is the energy stored \textbf{(i)} completely electrical (i.e., stored in the capacitor)? \textbf{(ii)} completely magnetic (i.e., stored in the inductor)?

\textbf{d)} At what times is the total energy shared equally between the inductor and the capacitor?

\textbf{e)} If a resistor is inserted in the circuit, how much energy is eventually dissipated as heat? \\
\hfill(NCERT-Physics 12.7 12Q)\\
\solution 
\pagebreak 

\item Obtain the resonant frequency and Q-factor of a series LCR circuit with $L = 3.0\, H$, $C = 27\, \mu F$, and $R = 7.4\, \Omega$. It is desired to improve the sharpness of the resonance of the circuit by reducing its `full width at half maximum' by a factor of 2. Suggest a suitable way.\\
\solution
\input{ncert-physics/12/7/21/Phy_12_7_21.tex}
\pagebreak
\item A radio can tune over the frequency range of a portion of the MW broadcast band: (800 kHz to 1200 kHz). If its LC circuit has an effective inductance (\(L\)) and a variable capacitor with capacitance (\(C\)), what must be the range of \(C\)?\\
\solution
\input{}
\pagebreak
\end{enumerate}
