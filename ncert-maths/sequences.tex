\begin{enumerate}[label=\thesection.\arabic*,ref=\thesection.\theenumi]
\item Find the number of terms in each of the following APs. 
\begin{enumerate}
    \item 7, 13, 19, ... 205

    \item 18, 15$\frac{1}{2}$, 13, ... -47
\end{enumerate}
\solution
\input{ncert-maths/10/5/2/5/ncert_10_5_2_5.tex}


\item For what value of $ n$, are the $ nth$ terms of two A.Ps: 63, 65, 67,\dots and 3, 10, 17,\dots equal?
\solution
\input{ncert-maths/10/5/2/15/asnmt2.tex}


\item Two APs have the same common difference.The difference between their $100${th} terms is 100,what is the difference between their $1000${th} terms?

\solution
\input{ncert-maths/10/5/2/12/1.tex}

\item Check whether -150 is a term of the AP: 11,8,5,2,....

 \solution
 \input{ncert-maths/10/5/2/6/question1.tex}
 

 \item Write the first five terms of the sequence \(a_n = \frac{n(n^2+5)}{4}\).

\solution
\input{ncert-maths/11/9/1/6/file1.tex}


\item
\begin{enumerate}
\item 30th term of the AP: 10, 7, 4, $\ldots$ is 
\item 11th term of the AP: $-3, -\frac{1}{2}, 2, \ldots$ is
\end{enumerate}
\solution
\input{ncert-maths/10/5/2/2/c.tex}


\item Write the first five terms of the sequence whose nth term is $\frac{2n-3}{6}$ and obtain the Z transform of the series
\solution
\input{ncert-maths/11/9/1/4/d.tex}

 \item For what values of x, the numbers $-\frac{2}{7}\,,x,-\frac{7}{2}\,$ are in G.P ?

\solution
\input{ncert-maths/11/9/3/6/assignment_11_9_3_6.Tex}


\item Find the $20^{th}$ and $n^{th}$ terms of the G.P $\frac{5}{2}$, $\frac{5}{4}$, $\frac{5}{8}$,.....

\solution
\input{ncert-maths/11/9/3/1/main.tex}


\item 
Which term of the following sequences:\\
(a) 2,$2\sqrt{2}$,4\dots is 128
\quad(b) $\sqrt{3}$,3,$3\sqrt{3}$\dots is 729\\
(c) $\frac{1}{3}$,$\frac{1}{9}$,$\frac{1}{27}$\dots is $\frac{1}{19683}$ \\
\solution
\input{ncert-maths/11/9/3/5/main.tex}
\clearpage

\item The number of bacteria in a certain culture doubles every hour. If there were 30 bacteria present in the culture originally, how many bacteria will be present at the end of $2^{nd}$ hour, $4^{th}$ hour and $n^{th}$ hour?

\solution
\input{ncert-maths/11/9/3/30/main.tex}


\item Ramkali saved Rs 5 in the first week of a year and then increased her weekly savings by Rs 1.75. If in the $n$th week, her weekly savings become Rs 20.75, find $n$.

\solution
\input{ncert-maths/10/5/2/20/main.tex}


\item Show that the sum of $\brak {m+n}^{th}$ and $\brak {m-n}^{th}$ terms of an $A.P.,$ is equal to twice the $m^{th}$ terms.    \\
\solution
\input{ncert-maths/11/9/5/1/assignment13.tex}


\item The sum of the first three terms of a G.P is $39/10$ and their product is $1$. Find the common ratio and the terms.\\
\solution
\input{ncert-maths/11/9/3/12/app.tex}


\item The ratio of the A.M and G.M of two positive numbers $a$ and $b$ is $m:n$. Show that $a:b = \brak{ m + \sqrt{m^2 - n^2}} : \brak{ m - \sqrt{m^2 - n^2}}$.\\
\solution
\input{ncert-maths/11/9/5/19/file2.tex}

\item The sum of three numbers in an arithmetic progression (AP) is $24$ and the product of those three numbers is $440$, find the values of the three numbers.\\
\solution

\item The sum of some terms of G.P. is $315$ whose first term and the common ratio are $5$ and $2$ , respectively. Find the last term and the number of terms.\\
\solution
\input{ncert-maths/10/5/3/20/discrete1.tex}

\item  Find the sum of n terms of the A.P. whose kth term is \(5k + 1\).\\
\solution

\item How many 3 digit numbers are divisible by 7? \\
\solution

\item A person writes a letter to four of his friends. He asks each one of them to copy the letter and mail to four different persons with instruction that they move the chain similarly. Assuming that the chain is not broken and that it costs 50 paise to mail one letter. Find the amount spent on the postage when 8th set of letter is mailed.\\
\solution 
\pagebreak

\item If $a$, $b$, $c$ are in A.P.; $b$, $c$, $d$ are in G.P and $\frac{1}{c}$, $\frac{1}{d}$, $\frac{1}{e}$ are in A.P. prove that $a$, $c$, $e$ are in G.P.\\
\solution
\pagebreak
\item Write the first five terms in the sequence:
\begin{align}
a_{0}  &= 3 \\
a_{n}  &= 3a_{n-1} + 2 \quad \text{for } n > 0
\end{align}
\solution
\pagebreak

\end{enumerate}
